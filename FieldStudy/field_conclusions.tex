\section{Field Conclusions}

The field study provided an excellent opportunity to bridge knowledge gained in Chapters 3 and 4 to a real-world study. Four streams in British Columbia were studied for the presence of several species of fish with a focus on Coho salmon. Using transects, researchers were able to obtain biomass measurements of each species in multiple sites throughout the streams, the results of which are summarized in Table~\ref{fig:kablemass}.

\vspace{5mm}

As seen in Table~\ref{fig:kablemass} Coho were only caught using conventional methods in stream BBB. Coho in stream BBB (Figure~\ref{fig:testBBBbiomCo}) were further confirmed to be present using eDNA methodology (Figure~\ref{fig:BBB_co}). The observed biomass of Coho in stream AAA was zero (Figure~\ref{fig:AAA_coho}), although Coho eDNA was detected at slight levels (Figure~\ref{fig:AAA_co}). The observed biomass of Coho in Stream DDD was also zero (Figure~\ref{fig:AAA_coho}), but eDNA analysis indicated the presence of Coho in 3 out of 6 sites at stream DDD. These detections may indicate that eDNA analysis is a more sensitive sampling approach and provides the possibility of detecting species which could not manually be confirmed. Overall, in the streams in which Coho was confirmed to be present using transects (stream BBB), we obtained 100 \% detection for Coho. Moreover, in this stream we obtained perfect sample replicate detection (8 out of 8).


\vspace{5mm}


In order to create statistical models for mean TCT of Coho we first created a simple linear model, model.co (Table~\ref{fig:modelco}) which already explained much of the variation in the data (evidenced by the $R^{2}$=0.6). This model only considered a single predictor, CO.Total.Biomass.g.  Plotting this regression line made the result clear, Coho biomass and mean TCT are highly correlated (Figure~\ref{fig:cohoanalysis}).

\vspace{5mm}

After fitting a simple linear model, we investigated a model in which numerous predictors were included, the so called `full model' (model.coho.full) as seen in Table~\ref{fig:fullcoho}. We chose to work with the predictor CO.Biomass.g.m3 as this quantity more closely resembles the density parameter discussed in Chapter 3. Using the full model as an initial model, we performed backward elimination. By considering a larger basis of possible predictors and their associated interactions in the full model, we were able to remove several predictors until our algorithm terminated on a final model. This model, model.co.step, summarized in Table~\ref{fig:cohostep}, ended up being a model that contained CO.Biomass.g.m3, as well as the interaction between Transect.Flow.cms and CO.biomass.g.m3 and the pH. In practice, we would include the main effect of Transect.Flow.cms in the model as well.

\vspace{5mm}


Model averaging further validated our stepwise model as the highest weighted model (with a weight of $w=0.35$) was the model containing those exact predictors. Table~\ref{fig:modelav} lists the associated term codes and components for the model average object. Best subset modelling provided further evidence in support of theses predictors, this, as the best performing model with three parameters as seen in Table~\ref{fig:cohobestsub} is the model containing CO.Biomass.g.m3, pH and the interaction between Transect.Flow.cms and CO.Biomass.g.m3.
 

\vspace{5mm}


For all fish, as seen in Table~\ref{fig:modelef}, our assumptions and testing failed to produce any meaningful results. Mean TCT and fish biomass showed no significant relationship.

\vspace{5mm}

In the end, multiple model selection techniques converged on the same few features for Coho. CO.Biomass.g.m3, pH, and the interaction between CO.Biomass.g.m3 and Transect.Flow.cms. We also include the main effect of Flow so that the interaction can be more easily interpreted. It appears that environmental covariates such as pH may indeed impact eDNA concentrations. Researchers wishing to obtain accurate and reliable TCT measurements, may wish to consider the impact of environmental covariates on their sampling routines.

