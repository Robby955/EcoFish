\startfirstchapter{Introduction to eDNA}
\label{chapter:introduction}





\hspace{4mm} 

The rapid and unfortunate decline in the Earth\textsc{\char13}s biodiversity is a clear and major obstacle facing humanity in the 21st century. Although this fact is widely accepted among the scientific community, the solution on how to combat this decline is still not known. One thing that is clear, however, is that researchers must expand and improve knowledge on the current state and distribution of biodiversity worldwide. In order to make informed and critical decisions regarding biodiversity, researchers must first monitor and model reliable distribution patterns, as well as estimate population sizes. Historically, this monitoring has involved invasive surveys that may even compound the issue of decline, especially when studying endangered or rare species. For this reason, it is critical that non-invasive, large scale biodiversity monitoring techniques are developed \citep{ednamonitoring}.


\vspace{5mm}


One such non-invasive technique for the monitoring of biodiversity involves the study and collection of environmental DNA. Environmental DNA, or eDNA, is mitochondrial or nuclear DNA that is released from an organism as it interacts with its environment.  Common sources of eDNA include shed skin cells, hair, blood, urine and mucous \citep{forestry}. Shed DNA can be collected by researchers for analysis. There are numerous methods in which eDNA is collected and analyzed, such as from samples of lake water \citep{goby} or other aquatic sources \citep{mussels}. eDNA has also been obtained from non-aquatic environments such as from snow, ice, or from soil \citep{soil}.

\newpage

 New and emerging technology has advanced researchers\textsc{\char13} ability to collect and analyze this shed eDNA. In particular, studies of eDNA have assisted scientists in detecting the extant of a species in a certain area and have facilitated conservation efforts \citep{speciesDetection,limitations}. Most commonly, environmental DNA is analyzed via DNA sequencing methods such as metagenomics and qPCR \citep{geneexpression}.



\vspace{5mm}




The analysis of eDNA allows researchers to study species without the capture of the target organism. This is of benefit in particular for species that are endangered and allows researchers to conduct investigations with minimal environmental disruption. Before the common usage of eDNA, methods for studying aquatic diversity included fishing and trapping. These invasive methods are expensive, time consuming and directly impact the habitat in which the researcher is studying. Indeed, using classical sampling methods to study fish may negatively impact the species and environment (especially electrofishing) \citep{electrofishing,freshwaterbiodiversity}. Collection and study of eDNA however, is non-invasive and only requires minimal sampling.\citep{ednaPH,streamdwelling}. Analysis of environmental DNA has already been used successfully as a surveillance method for rare fish species \citep{rareaquatic}.


\vspace{5mm}

The ability to sequence miniscule concentrations of eDNA directly from aquatic environments is revolutionizing the field of ecological monitoring and management \citep{MarineSystems}. Although most studies regarding eDNA in aquatic environments have focused on freshwater, eDNA is increasingly being used to study ocean and marine systems.  Corporations involved with aquatic conservation and fisheries management are progressively incorporating eDNA into their business and research \citep{usesofedna}. Hence, it is critical that researchers understand the physical and chemical properties of eDNA, such as degradation or the impact of ecological covariates. The research we present in this thesis has the potential to be of particular use for those interested in fisheries monitoring. 


   \vspace{4mm}


One species that researchers have studied using eDNA methodology is the invasive American signal crayfish, \textit{Pacifastacus leniusculuscray}. This crayfish is problematic and is the current leading cause of decline among UK\textsc{\char13}s native crayfish species. The practice of eDNA related techniques allows for non-invasive and potentially early detection of these crayfish \textit{Pacifastacus leniusculuscray}. Early detection of eDNA may allow for their eradication before the levels are large enough to detect manually and before it is too late to act \citep{crayfish}.

    
\vspace{5mm}


Researchers have also used eDNA to study the Coastal tailed frog (\textit{Ascaphus truei}). By using eDNA and testing water samples at a variety of lakes and watersheds, researchers were able to expand knowledge of known Coastal tailed frog distributions in British Columbia \citep{forestry}.

    \vspace{5mm}

The association between eDNA concentration and species abundance is the subject of ongoing study. Research of native fish populations have shown that studying eDNA concentration in water samples may be able to predict fish abundance as well as, or better than more invasive and costly methods \citep{fishabundance,salmonabundance}. Moreover, an ever-increasing number of non-invasive eDNA techniques produce reliable enough DNA to address nearly all questions that could be obtained via traditional methods \citep{ecounderstanding}.
\vspace{5mm}

In Florida, researchers collected eDNA to study the abundance and occupancy of invasive Burmese Pythons, \textit{Python molurus bivitatus}, in the Everglades \citep{pythons}. These pythons pose significant threats to native species.  Burmese Pythons are considered to  be largely mysterious and their population distribution in the Florida area is not well documented. These pythons mainly reside within inaccessible habitats, making standard sampling nearly impossible. Because they are so evasive, only a single study with the goal of estimating detection probability has been conducted in which standard sampling techniques were attempted \citep{standardPython}.  This standard study did not have very promising results and had very few sightings of the pythons. However, researchers using eDNA sampling were able to extensively expand knowledge of the python distribution via studies done on water samples taken throughout the Everglades. 

 \vspace{5mm}
Overall, the study and analysis of environmental DNA is an emerging method of science that has been used successfully for the detection and study of rare species. The less invasive methods utilized while studying eDNA are becoming increasingly important as biodiversity continues to decline worldwide \citep{GreenSturgeon}.

\vspace{5mm}

 In Chapter 2 of this thesis, we introduce the preliminary content that is needed to understand the experiments. This includes a survey of methods used to analyze eDNA, and a summary of past research on the topic of eDNA. In Chapter 3 of this thesis, we use water samples from tanks in which the biomass of Coho is known to study the relationship between eDNA concentration and biomass \citep{fishforensics}. In Chapter 4, we extend this result by considering the impact of flow or dilution on eDNA. The experiments in Chapters 3 and 4 were conducted by Jeff MacAdams in collaboration with EcoFish Research. In Chapter 5, we bridge the gap to the field, where we build models for predicting Coho biomass using eDNA concentrations and several environmental covariates (collected by EcoFish personnel). Finally, in Chapter 6 we provide a summary of the thesis and present our conclusions.





