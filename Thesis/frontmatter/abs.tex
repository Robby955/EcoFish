\newpage
\TOCadd{Abstract}



\begin{center}
\textbf{ABSTRACT}
\end{center}


Standard sampling and monitoring of fish populations are invasive and time-consuming techniques. The ongoing development of statistical techniques to analyze environmental DNA (eDNA) introduces a possible solution to these challenges. We analyzed and created statistical models for qPCR data obtained from two controlled experiments that were conducted on samples of Coho salmon at the Goldstream Hatchery.

\vspace{5mm}

The first experiment analyzed was a density experiment whereby varying numbers of Coho (1, 2, 4, 8, 16, 32 and 65 fish) were placed in separate tanks and eDNA measurements were taken. The second experiment dealt with dilution, whereby three Coho were placed into tanks, removed and eDNA was then sampled at dilution volumes of 20kL, 40kL, 80kL, 160kL and 1000kL.

\vspace{5mm}

Finally, we analyzed a set of field data from several streams in the Pacific North West for the presence of Coho salmon. In the field models, we considered the impact of environmental covariates as well as eDNA concentrations.

\vspace{5mm}

 Our analysis suggests that eDNA concentration can be used as a reliable proxy to estimate Coho biomass. 
